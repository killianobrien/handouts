\documentclass[oneside,10pt]{amsart}
\usepackage{jfkobgrp}

\begin{document}
\tuttitle{Weeks 16--18 $|$ Group Actions}{Dr Killian O'Brien \& Dr Jan Foniok}{6GZ3012 Group Theory}{\today}

\topic{Reading notes on Chapter 14: Group Actions}

The notion of a \emph{group action} captures the phenomenon of a group representing some
permutations or transformations of some set.
For example, a dihedral group can be seen as the permutations of a regular $n$-gon, or of its vertices;
or the general linear group~$GL_2(\mathbb{R})$ consists of matrices that represent linear transformations
of the Euclidean plane~$\mathbb{R}^2$.

Our textbook takes the view that the most important cases are
actions of the group on itself:
either by left/right multiplication, or by \emph{conjugation}.
These actions can be used to prove \emph{Burnside's Counting Theorem},
which provides a way to count permutations up to some equivalence, etc.,
and then later in Chapter~15 we can use these resutls to prove the Sylow theorems,
the crux and highlight of our Group Theory course.


\topic{Chapter 14 exercises}

Exercises 1.--5.\ relate to Section 14.1;
exercises 6.--7.\ to Section~14.2;
exercises 8.--19.\ to Section~14.3;
and exercises 20.--25.\ are extensions.

The Sage exercises deal with groups of \emph{automorphisms of graphs},
which I find personally very interesting to explore.

\topic{Submission problems}

Let $G$ be a group acting on a set~$X$.
Prove that for any fixed element~$g$ of~$G$,
the mapping $x \mapsto gx$ is a permutation of the set~$X$.

Exercises 4 and 13 on page 236.

\end{document}
