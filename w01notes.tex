\documentclass[oneside,10pt]{amsart}
\usepackage[paper=a4paper,top=3cm, bottom=2cm,left=2cm,right=3cm]{geometry}
\usepackage{setspace}
\usepackage{xcolor}
\usepackage{graphicx}
\usepackage{sidenotes}
\usepackage{hyperref}
\usepackage{sagetex}
\usepackage{mathtools}
\usepackage{tcolorbox}
\usepackage[ddmmyy]{datetime}
\setlength{\parindent}{0cm}
\setlength\itemsep{3em}

%a background shaded textbox with white text. spans the width of the textarea
\newcommand{\cbox}[1]{\begin{tcolorbox}[left=0.5mm,right=0.5mm,top=0.5mm,bottom=0.5mm, boxsep=2pt, boxrule=0pt,colback=black]\color{white}\sffamily #1 \end{tcolorbox}}

%sheet title banner, uses \cbox, 4 arguments: top left, top right, bottom left, bottom right
\newcommand{\tuttitle}[4]{\cbox{\textbf{#1} \hfill {#2}\\{#3} \hfill {\Small #4}}\vskip 4pt}

%topic banner
\newcommand{\topic}[1]{\cbox{\textbf{#1}}\vskip 4pt}

%subtopic banner
\newcommand{\subtopic}[1]{\tcbox[nobeforeafter, left=0mm,right=10mm,top=0.5mm,bottom=0mm, boxsep=2pt, boxrule=0pt,colback=black!50]{\color{white}\sffamily #1}\\}

\begin{document}
\tuttitle{Weeks 0 \& 1 $|$ Introduction \& preliminaries}{Dr Killian O'Brien}{6GZ3012 Group Theory}{\today}

\topic{Welcome and introduction}
\begin{itemize}
\item
Schedule, teaching pattern and labs/tutorials
\item
Assessment
\begin{itemize}
\item
Coursework (40\%)
\item
Examination, 3 hours (60\%)
\end{itemize}
\item
The textbook: \textit{Abstract Algebra: Theory and Applications}, by Tom Judson and Rob Beezer
\item
Quick tour of the Moodle area
\item
Lecture recordings
\end{itemize}
\topic{Reading notes on chapters 1 \& 2}
\textbf{Chapter 1}\\
This chapter should be a nice read for you as it reviews material on sets, mappings and equivalence relations that should be familiar to you from your study in your first and second year. Some topics to look out for and study well are
\begin{itemize}
\item
The Cartesian product of sets
\item
Mapping/function concepts of well-defined, injective (one-to-one),  surjective (onto), bijective, inverses, ...
\item
Theorem 1.15
\item
Theorem 1.20
\item
Equivalence relations and partitions
\item
Theorem 1.25
\end{itemize}

\textbf{Chapter 2.}\\
You will recognise the material in this chapter from our work in the Number Theory and Cryptography unit last year. It runs through the key results that we will make use of in the study of group theory. There are lots of good exercises in section 2.3 focusing mainly on the use of the proof by induction technique and the concepts of divisibility and prime numbers.


\topic{Problem workshop}
There are lots of good refresher exercises in chapters 1 and 2 and you should attempt as many as you can. We will consider some of the following in lectures
\begin{itemize}
\item
Chapter 1. Exercises (Sec. 1.3) questions 11, 17, 19, 25, 28
\item
Chapter 2. Example 2.4, Exercises (Sec. 2.3) questions 4, 6, 7
\end{itemize}

\topic{Submission problems}
Throughout the unit we will ask you to submit your treatments of various problems from the textbook or elsewhere. We will scrutinise these submissions and give you formative feedback on them. This will help to structure some of your self-study time and help you to improve your mathematical writing skills. Some of these problems may make an appearance on the summative coursework later in the unit.
\begin{itemize}
\item
Chapter 2. Exercises 2.3 Questions 10 and 17
\end{itemize}
Please submit these questions to Killian by end of Friday 6th October.

\end{document}