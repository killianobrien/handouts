\documentclass[oneside,10pt]{amsart}
\usepackage{enumitem}
\usepackage{jfkobgrp}
%computer icon
\newcommand{\comp}{\begin{marginfigure}\includegraphics[scale=0.08]{comp.png}\end{marginfigure}}
\newcommand{\logo}{\marginpar{\vskip -1mm \includegraphics[scale=0.1]{mmulogo.png}}}

\newcounter{pass}

%%%%Headers
%\usepackage{fancyhdr}
%\rhead{}
%\rfoot{}
%\setlength{\headheight}{0pt}
%\setlength{\textwidth}{390pt}
%\pagestyle{fancy}

%lfoot{\texttt{6G5Z3006 2016/17}}
%rfoot{Number Theory \& Cryptography}
%cfoot{\thepage}
%%%%%%%%

%%%% no indent on new paragraphs
\setlength{\parindent}{0pt}
\setlength{\parskip}{3pt}

%%%% command for question part marks
\newcommand{\qmarks}[1]{\begin{flushright} [ #1 ] \end{flushright}}

\pagestyle{plain}

\begin{document}
%\logo
\tuttitle{6G6Z3012 Group Theory}{Dr Jan Foniok \& Dr Killian O'Brien}{Coursework 2017/18}{\today}

This is the Coursework element of assessment for the \texttt{6G6Z3012} unit. Please read all these instructions carefully.

\topic{Your report}
You will need to prepare a report detailing your solutions of the attached problems. Reports can be word-processed or handwritten, but in either case your arguments should be clearly explained and justified with reference to results from the unit. We recommend typesetting the report with \LaTeX\ as this allows you to produce professional output and makes re-drafting of your work easier.

\topic{Submission and return schedule}
The due date for submission is {\bfseries Friday~19th~January~2018}. You will need to download and complete your barcoded cover sheet and submit your coursework to the Coursework Receipting Office by posting it into one of the blue submission postboxes located in the John Dalton Building.

Further details on submission procedures can be found at \url{coursework.mmu.ac.uk} or via the Assessments block on the unit's Moodle area.

Marks will be uploaded to the Moodle area and the marked courseworks with feedback ready for return within the university's 4 week return period.

\topic{Marking guidance}
The coursework contributes 40\% to your overall mark for the unit, the remaining 60\% coming from the exam. Part marks shown in the questions indicate the relative weighting of each question.

Marks will be awarded to reports according to how well they present correct treatments of the questions, where the reasoning is clearly explained and justified, and accompanied by appropriate references to the concepts and results from the unit. You should attempt all questions. Marks will be awarded for partially correct solutions

Some questions ask you to use Sage to perform various tasks. You should include a printout of the CoCalc worksheet(s) which include the necessary Sage code and output along with some explanatory comments as appropriate.

\textbf{Read each question carefully and ensure that your report follows the guidance given and supplies all the requested information.}

\topic{Learning outcomes}
The learning outcomes of the unit assessed in this coursework assert that a student successfully completing this unit will be able to:
\begin{itemize}
\item
recognize, discuss and calculate with a wide variety of groups;
\item
construct rigorous arguments to solve problems in group theory;
\item
present expositions of important aspects of the theory of groups;
\item
use mathematical software to analyze problems in group theory.
\end{itemize}

\topic{Feedback}
Formative feedback that supports your work on the coursework takes place in the weekly class sessions. The marked courseworks will be returned with personalized feedback provided on your work. Full model solutions will also be provided.

\topic{Office hours and support}
The regular drop-in office hours for the staff are
\begin{center}
\begin{tabular}{rl}
Dr Killian O'Brien:  & Mon 14:30 -- 15:30, Tues 11:00 -- 13:00,\\
Dr Jan Foniok:  & Thur 15:00 -- 17:00, Fridays 10:00 -- 11:00.
\end{tabular}
\end{center}

To arrange a meeting please contact us on \texttt{j.foniok@mmu.ac.uk} or \texttt{k.m.obrien@mmu.ac.uk}.

\topic{Questions}
Some of the questions here are taken from Judson's \textit{Abstract Algebra: Theory and Applications} as shown in the parantheses. Each question begins with some commentary on the background and any necessary definitions.

\topic{1. Subgroups of dihedral groups \hfill 25 marks}
Let $n$ be a positive integer greater than 2. The dihedral group $D_n$ is the symmetry group of a regular $n$-sided polygon centred at the origin. It is generated by a rotation, $r$, counter-clockwise about the origin through an angle $2\pi/n$, and a reflection $s$, in an axis running through one of the polygon's vertices and the origin. These generators for $D_n$ are subject to the relations $r^n = e$, $s^2 = e$ and $sr = r^{-1}s$. Each element of $D_n$ can be expressed in the standard form $r^i s^j$, where $0 \leq i \leq n-1$ and $j=0$ or $1$.

\noindent
A group $G$ is \emph{Lagrangian} if for every positive divisor $d$ of $|G|$ there is a subgroup $H$ of $G$ with $|H| = d$.
\begin{enumerate}[label=(\alph*)]
\item
Give a complete description of the subgroups of $D_n$ for all $n \geq 2$. Your description should detail the elements of each subgroup, prove that they are subgroups and prove that there are no other subgroups apart form the ones you describe.
\item
Explain how your treatment confirms the fact that the dihedral groups are Lagrangian.
\item
Which of the subgroups are normal in $D_n$?
\end{enumerate}
\topic{2. The tetrahedron (AATA Sec. 5.3 Q. 16, Sec. 5.5 Q. 3) \hfill 25 marks}
A regular tetrahedron is a polyhedron whose four faces are congruent regular triangles.
\begin{enumerate}[label=(\alph*)]
\item
Describe the group, $G$, of rotational symmetries of the regular tetrahedron.
\item
By using an appropriate labelling of parts of the tetrahedron prove that $G$ is isomorphic to $A_4$, the alternating group on four symbols.
\item
Construct the group of symmetries of the tetrahedron in Sage (also the alternating group on 4 symbols, \(A_4\)) with the command \texttt{A=AlternatingGroup(4)}.  Using tools such as orders of elements, and generators of subgroups, see if you can find \emph{all of} the subgroups of \(A_4\) (each one exactly once).  Do this without using the \texttt{.subgroups()} method to justify the correctness of your answer (though it might be a convenient way to check your work).%

For each of the subgroups, determine whether it is normal in~$A_4$ or not, and justify your answer.

Provide a nice summary as your answer~-- not just piles of output.  So use Sage as a tool, as needed, but basically your answer will be a concise paragraph and/or table.
\end{enumerate}
\newpage
\topic{3. Classifying groups of order $2p$ (AATA Sec. 9.3 Q. 55) \hfill 30 marks}
Write an account of the classification of the isomorphism classes of all groups of order $2p$, where $p$ is prime. Your arguments can follow the steps outlined below, but you should present your account as a coherent standalone piece, with references to results from AATA as needed.

Assume \(G\) is a group of order \(2p\), where \(p\) is an odd prime.
\begin{itemize}
\item\hypertarget{li-370}{}If \(a \in G\), show that \(a\) must have order 1, 2, \(p\), or \(2p\).%
\item\hypertarget{li-371}{}Suppose that \(G\) has an element of order \(2p\).  Prove that \(G\) is isomorphic to \({\mathbb Z}_{2p}\).  Hence, \(G\) is cyclic.%
\item\hypertarget{li-372}{}Suppose that \(G\) does not contain an element of order \(2p\).  Show that \(G\) must contain an element of order \(p\).  {\em Hint}:  Assume that \(G\) does not contain an element of order \(p\).%
\item\hypertarget{li-373}{}Suppose that \(G\) does not contain an element of order \(2p\).  Show that \(G\) must contain an element of order 2.%
\item\hypertarget{li-374}{}Let \(P\) be a subgroup of \(G\) with order \(p\) and \(y \in G\) have order 2.  Show that \(yP = Py\).%
\item\hypertarget{li-375}{}Suppose that \(G\) does not contain an element of order \(2p\) and \(P = \langle z \rangle\) is a subgroup of order \(p\) generated by \(z\).  If \(y\) is an element of order 2, then \(yz = z^ky\) for some \(2 \leq k < p\).%
\item\hypertarget{li-376}{}Suppose that \(G\) does not contain an element of order \(2p\).  Prove that \(G\) is not abelian.%
\item\hypertarget{li-377}{}Suppose that \(G\) does not contain an element of order \(2p\) and \(P = \langle z \rangle\) is a subgroup of order \(p\) generated by \(z\) and \(y\) is an element of order 2. Show that we can list the elements of \(G\) as \(\{z^iy^j\mid 0\leq i < p, 0\leq j < 2\}\).%
\item\hypertarget{li-378}{}Suppose that \(G\) does not contain an element of order \(2p\) and \(P = \langle z \rangle\) is a subgroup of order \(p\) generated by \(z\) and \(y\) is an element of order 2.  Prove that the product \((z^iy^j)(z^ry^s)\) can be expressed uniquely as \(z^m y^n\) for some non negative integers \(m, n\).  Thus, conclude that there is only one possibility for a non-abelian group of order \(2p\), it must therefore be the one we have seen already, the dihedral group.%
\end{itemize}

\topic{4. The importance of normal subgroups and factor groups \hfill 20 marks}
Normal subgroups and the factor group construction provide a way to get a simplified view of a group $G$ by partitioning its elements into subsets and looking at the operation induced on the partition by the operation from $G$. In this question you will prove that normal subgroups are the only such way to obtain simplified views of $G$.

A \emph{congruence} on a group $G$ is an equivalence relation $\sim$ on $G$ that is compatible with the group operation of $G$, that is, if $g_1 \sim g_2$ and $h_1 \sim h_2$ then $g_1h_1 \sim g_2h_2$.

Let $\sim$ be a congruence on~$G$.
\begin{enumerate}[label=(\alph*)]
\item
Prove that if $g_1 \sim g_2$ then $g_1^{-1} \sim g_2^{-1}$.
\item
Prove that the partition of $G$ induced by the equivalence classes of $\sim$ is the partition of $G$ into the cosets of a certain normal subgroup of $G$.
\end{enumerate}





\end{document}

%%%%%%%%%%%%%%%%%%%%%%%%%
%sagemathcloud={"latex_command":"pdflatex -synctex=1 -interact=nonstopmode 'cwk.tex'"}
