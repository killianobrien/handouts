\documentclass[oneside,10pt]{amsart}
\usepackage{jfkobgrp}

\begin{document}
\tuttitle{Week 6  $|$ Lagrange's theorem}{Dr Killian O'Brien \& Dr Jan Foniok}{6GZ3012 Group Theory}{\today}

\topic{Reading notes on Chapter 6: Cosets and Lagrange's Theorem}

In chapter 4 on cyclic groups you proved in theorem 4.10 that every subgroup of a cyclic group is cyclic (i.e. generated by a single element) and moreover, in theorem 4.13, that the order of any element of a cyclic group divides the order of the group.

This latter result generalises to the famous Lagrange's Theorem, which says that in any finite group $G$, the order of any subgroup of $G$ is always a divisor of the order of $G$.

This chapter uses the concept of \emph{cosets} (a type of `translate' of a subgroup) to prove the theorem.

It is important to be aware of the status of what is commonly known as the converse of Lagrange's theorem, i.e. is there a subgroup of $G$ or order $d$ is $d$ is a divisor of the order of $G$? It is not necessary that there is. Proposition 6.15 shows a counter-example to this using the group $A_4$.

Section 6.3 then looks at some applciations of Lagrange's theorem to the multiplicative group of integers modulo $n$. These results will be familiar to you from your study of Number Theory last year.

\topic{Submission problems}
Questions 6, 11 and 19 from section 6.4.

We may show some of the parts of question 11 in the lectures. Question 6 requires some thought and investigation but question 19 should be relatively straightforward compared to the other two.

\end{document}
