\documentclass[oneside,10pt]{amsart}
\usepackage[paper=a4paper,top=3cm, bottom=2cm,left=2cm,right=3cm]{geometry}
\usepackage{setspace}
\usepackage{xcolor}
\usepackage{graphicx}
\usepackage{sidenotes}
\usepackage{hyperref}
%\usepackage{sagetex}
\usepackage{mathtools}
\usepackage{tcolorbox}
\usepackage{enumitem}
\usepackage[ddmmyy]{datetime}
\setlength{\parindent}{0cm}
\setlength\itemsep{3em}

%a background shaded textbox with white text. spans the width of the textarea
\newcommand{\cbox}[1]{\begin{tcolorbox}[left=0.5mm,right=0.5mm,top=0.5mm,bottom=0.5mm, boxsep=2pt, boxrule=0pt,colback=black]\color{white}\sffamily #1 \end{tcolorbox}}

%sheet title banner, uses \cbox, 4 arguments: top left, top right, bottom left, bottom right
\newcommand{\tuttitle}[4]{\cbox{\textbf{#1} \hfill {#2}\\{#3} \hfill {\Small #4}}\vskip 4pt}

%topic banner
\newcommand{\topic}[1]{\cbox{\textbf{#1}}\vskip 4pt}

%subtopic banner
\newcommand{\subtopic}[1]{\tcbox[nobeforeafter, left=0mm,right=10mm,top=0.5mm,bottom=0mm, boxsep=2pt, boxrule=0pt,colback=black!50]{\color{white}\sffamily #1}\\}

\begin{document}
\tuttitle{Weeks 0 \& 1 $|$ Feedback on submission problems}{Dr Killian O'Brien}{6GZ3012 Group Theory}{\today}

Individual feedback has been given to you on your submissions. The problems served as a nice way to reacquaint ourselves with proof-by-induction -- though some of the details of handling the Fibonacci sequence can be tricky. Here are some comments on the parts of question 17.

\medskip

The Fibonacci numbers are%
\begin{equation*}
1, 1, 2, 3, 5, 8, 13, 21, \ldots.
\end{equation*}
We can define them inductively by \(f_1 = 1\), \(f_2 = 1\), and \(f_{n + 2} = f_{n + 1} + f_n\) for \(n \in {\mathbb N}\). \leavevmode%
\begin{enumerate}[label=(\alph*)]
\item Prove that \(f_n < 2^n\).%
\item Prove that \(f_{n + 1} f_{n - 1} = f^2_n + (-1)^n\), \(n \geq 2\).%
\item Prove that \(f_n = [(1 + \sqrt{5}\, )^n - (1 - \sqrt{5}\, )^n]/ 2^n \sqrt{5}\).%
\item Show that \(\lim_{n \rightarrow \infty} f_n / f_{n + 1} = (\sqrt{5} - 1)/2\).%
\item Prove that \(f_n\) and \(f_{n + 1}\) are relatively prime.%
\end{enumerate}

\begin{enumerate}[label=(\alph*)]
\item
\end{enumerate}

\end{document}