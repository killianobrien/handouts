\documentclass[oneside,10pt]{amsart}
\usepackage[paper=a4paper,top=3cm, bottom=3cm,left=2cm,right=3cm]{geometry}
\usepackage{setspace}
\usepackage{xcolor}
\usepackage{graphicx}
\usepackage{sidenotes}
\usepackage{hyperref}
\usepackage{sagetex}
\usepackage{mathtools}
\setlength{\parindent}{0cm}

%a background shaded textbox with white text. spans the width of the textarea
\newcommand{\cbox}[1]{\colorbox{black}{\begin{minipage}{\linewidth}\color{white}\sffamily #1 \end{minipage}}}

% gray background box
\newcommand{\gbox}[1]{\colorbox{gray!30}{\begin{minipage}{\linewidth}\color{black} #1 \end{minipage}}}

%sheet title banner, uses \cbox
\newcommand{\tuttitle}[4]{\cbox{\textbf{#1} \hfill {#2}\\{#3} \hfill {\Small #4}}\vskip 4pt}

%topic banner
\newcommand{\topic}[1]{\cbox{\textbf{#1}}\vskip 4pt}


\newcounter{pass}





\setlength\itemsep{3em}
\begin{document}
\tuttitle{Weeks 03, 04, 05 $|$ Initial group theory \& examples}{Dr Killian O'Brien}{6GZ3012 Group Theory}{\today}
\vskip 16pt

\topic{Reading notes on Chapter 4: Cyclic Groups}
This short chapter deals with the type of group structures (groups and subgroups within larger groups) than can be generated by a single group element.

Let $g$ be an element of a group $G$. The subgroup (although it may be all of $G$) generated by $g$ is the set of all elements of $G$ of the form $g^n$ where $n \in \mathbb{Z}$. The notation for this subgroup is $\langle g \rangle$, i.e.
$$\langle g \rangle = \left \{ g^n \, : \, n \in \mathbb{Z} \right \}.$$ Such (sub)groups are referred to as cyclic (sub)groups.

The chapter introduces this concept, gives some examples, in particular focussing on on the multiplicative group of complex numbers and the roots of unity.

The most signinficant result in the chapter might be Theorem 4.10 which establishes that every subgroup of a cyclic group is in fact cyclic. This prove uses the concepts of integer division with remainder and the well orderedness of the integers, concepts familiar to you from your Number Theory studies last year. The theorems following 4.10 give further applications of these number theoretic concepts to cyclic groups.

\vskip 4pt

\topic{Chapter 4 exercises}
Section 4.4 contains a large number of useful exercises.

\begin{itemize}
\item
Questions 1 through 22 focus mainly on examples. Many of these exercises concern the complex numbers and number theory type exercises which should be familiar to you.
\item
From question 23 on we find questions of a more abstract group-theoretic nature.
\end{itemize}

\vskip 4pt

\topic{Sage exercises}
Section 4.8 is a guided exploration of what Judson refers to as $U(n)$, the group of units (elements with multiplicative inverses) from the additive group $\mathbb{Z}_n$. This is the same as the group that we dealt with in Number Theory last year which we referred to as $\mathbb{Z}_n^\times$.

\end{document}